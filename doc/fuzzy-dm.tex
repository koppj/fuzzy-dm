\documentclass{scrartcl}

\usepackage{amsmath,amssymb,slashed}
\usepackage{cleveref}
\usepackage{graphicx}
\usepackage{ctable}
\usepackage{longtable}
\usepackage{underscore}

% Define bold face typewriter font
\DeclareFontShape{OT1}{cmtt}{bx}{n}{
  <5><6><7><8><9><10><10.95><12><14.4><17.28><20.74><24.88>cmttb10}{}


\title{Couplings between neutrinos and fuzzy dark matter in GLoBES}
\author{Vedran Brdar, Joachim Kopp, Jia Liu, Xiaoping Wang}
\date{v1.0 (June 2019)}

\begin{document}

\maketitle

\section{Introduction}
%---------------------

If the dark matter (DM) in the Universe couples to neutrinos, the propagation
of neutrinos through the omnipresent DM background can affect oscillations.
This is similar in spirit to Mikheyev-Smirnov-Wolfenstein-type matter effects
in the Standard Model: neutrinos can undergo coherent forward scattering on
the background particles, i.e.\ they can interact with the background
without changing their quantum state, or the background's quantum state. It
is therefore impossible to tell which background particle a given neutrino
has interacted with, so all of them contribute coherently, leading to
a huge enhancement factor. The enhancement factor is proportional to the
number density of backgroud particles, and it is for this reason that
neutrino--DM interactions are most easily observable for extremely light DM
particles ($\lesssim 10^{-20}$\,eV), which are called ``fuzzy DM'' because of
their huge Compton wave length. Given the known mass density of DM,
fuzzy DM must have a huge number density.  Neutrino interactions with fuzzy DM
have first been investigated in \cite{Berlin:2016woy, Krnjaic:2017zlz,
Brdar:2017kbt}.

The present code makes this scenario accessible to GLoBES.
It supports both, couplings of neutrinos to fuzzy DM in the form of new
scalar particles, or in the form of new vector bosons. In the latter case,
it is possible to specify the polarization of the DM relative to the neutrio
beam direction, or to assume unpolarized DM.  (It is not fully understood
whether or not fuzzy vector DM would be polarized or unpolarized in the Milky
Way.)


\subsection{Scalar DM}

For scalar DM, the relevant terms in the phenomenological Lagrangian are
\cite{Brdar:2017kbt}
\begin{align}
  \mathcal{L}_\text{scalar}
    &= \bar\nu_L^\alpha i \slashed\partial \nu_L^\alpha
     - \tfrac{1}{2} m_\nu^{\alpha\beta} \overline{(\nu_L^c)^\alpha} \nu_L^\beta
     - \tfrac{1}{2} y^{\alpha\beta} \phi \, \overline{(\nu_L^c)^\alpha} \nu_L^\beta \,.
  \label{eq:L-scalar}
\end{align}
The first term is the standard kinetic term, the second one is a standard
Majoarana mass term, and the third one describes the couplings between
the scalar DM particle $\phi$ and the left-handed SM neutrinos
$\nu_L^\alpha$. The strength of the coupling is denoted by the complex symmetric
matrix $y^{\alpha\beta}$.
In all terms, the flavor indices $\alpha$ and $\beta$
run over $e$, $\mu$, or $\tau$, and $\nu_L^c$ refers to the charge conjugated
field.

The DM field $\phi$ in \cref{eq:L-scalar} can be written as
\begin{align}
  \phi = \phi_0 \cos(m_\phi t) \,,
  \label{eq:phi-osc}
\end{align}
with the DM mass $m_\phi$ and the normalization
\begin{align}
  \phi_0 = \frac{\sqrt{2 \rho_\phi}}{m_\phi} \,.
  \label{eq:phi-0}
\end{align}
In the last expression, $\rho_\phi \simeq 0.3\,\text{GeV}/\text{cm}^3$ is the
DM energy density at the location of the Earth ($\sim 8$~\,kpc from the
Galactic Center). \Cref{eq:phi-osc} describes a slowly varying
function of time, with a periodicity $T = 2\pi/m_\phi = 4.77\,\text{days}
\times (10^{-20}\,\text{eV} / m_\phi)$.

In view of \cref{eq:phi-0}, we define the \emph{effective DM couplings}
\begin{align}
  \chi^{\alpha\beta} \equiv y^{\alpha\beta} \frac{\sqrt{2 \rho_\phi}}{m_\phi} \,.
  \label{eq:chi-dm}
\end{align}

We see from \cref{eq:L-scalar} that neutrino couplings to scalar DM
can be viewed as a dynamic modification to the neutrino mass matrix.
This is exactly how they are implemented in {\tt fuzzy-dm.c}: the code
reconstructs the vacuum mass matrix from the given parameters, adds
the DM contribution, and then re-diagonalizes the result to obtain an
effective PMNS matrix and effective mass squared differences.


\subsection{Vector DM}

The Lagrangian for DM couplings to fuzzy vector DM is \cite{Brdar:2017kbt}
\begin{align}
  \mathcal{L}_\text{vector}
    &= \bar\nu_L^\alpha i \slashed\partial \nu_L^\alpha
     - \tfrac{1}{2} m_\nu^{\alpha\beta} \overline{(\nu_L^c)^\alpha} \nu_L^\beta
     + g Q^{\alpha\beta} \phi^\mu \bar{\nu}_L^\alpha \gamma_\mu \nu_L^\beta \,,
  \label{eq:L-vector}
\end{align}
The first two terms, which appear also in the standard scenario,  are the
same as in \cref{eq:L-scalar} above; the last term has the form of a gauge
coupling with gauge boson $\phi^\mu$ and flavor-dependent (complex hermitian)
charges $Q^{\alpha\beta}$.  We write $\phi^\mu$ as
\begin{align}
  \phi^\mu = \phi_0 \xi^\mu \cos(m_\phi t) \,,
  \label{eq:phi-osc-vector}
\end{align}
where $\phi^0$ is the same as in \cref{eq:phi-0}.  The polarization vector $\xi^0$
describes the alignment of the DM field relative to the beam, where the
beam is always assumed to travel in the positive $z$-direction.
The effect of the DM--neutrino coupling for vector DM is to modify the neutrino
dispersion relation into
\begin{align}
  E^2 = p^2 + m_\nu^2  \qquad\to\qquad
  E^2 &= [p^\mu + g Q \xi^\mu \phi_0]^2 + m_\nu^2 \,.
  \label{eq:dispersion}
\end{align}
(Here, we have for simplicity dropped the time-dependent cosine functions.)
The code uses \cref{eq:dispersion} to compute the modified Hamiltonian for
neutrinos propagating through the DM background.  Specifically, the Hamiltonian
turns into
\begin{align}
  \hat{H} = \frac{1}{2E} U^\dag \begin{pmatrix}
                                  0 &                 &                  \\
                                    & \Delta m_{21}^2 &                  \\
                                    &                 & \Delta m_{31}^2
                                \end{pmatrix}  U
          + g^2 \phi_0^2
            \begin{pmatrix}
              |Q^{ee}|^2     &  |Q^{e\mu}|^2    &  |Q^{e\tau}|^2    \\
              |Q^{\mu e}|^2  &  |Q^{\mu\mu}|^2  &  |Q^{\mu\tau}|^2  \\
              |Q^{\tau e}|^2 &  |Q^{\tau\mu}|^2 &  |Q^{\tau\tau}|^2
            \end{pmatrix}
          + g Q \phi_0 p^\mu \xi_\mu \,.
\end{align}
The last term is relevant only for polarized vector DM -- for unpolarized DM,
it averages to zero when $\xi^\mu$ is averaged over.


\section{Initializing the Oscillation Engine}
%--------------------------------------------

To use the code, include {\tt fuzzy-dm.c} in your project by modifying your
{\tt Makefile} accordingly, and {\tt \#include} the header file {\tt fuzzy-dm.h}
in your source code.

The first step is to initialize the oscillation engine using the function
\begin{verbatim}
  int dm_init_probability_engine(int dm_type);
\end{verbatim}
The argument {\tt dm_type} can be {\tt DM_SCALAR} for scalar DM,
{\tt DM_VECTOR_POLARIZED} for polarized vector DM, or
{\tt DM_VECTOR_UNPOLARIZED} for unpolarized vector DM.  These three
constants are defined in {\tt fuzzy-dm.h}.  The DM type can be
changed later using
\begin{verbatim}
  int dm_set_dm_type(int dm_type);
\end{verbatim}
and it can be queried using
\begin{verbatim}
  int dm_get_dm_type();
\end{verbatim}

The oscillation engine knows of course the six standard oscillation parameters
(named {\tt TH12}, {\tt TH13}, {\tt TH23}, {\tt DELTA_0}, {\tt DM21}, and
{\tt DM31}), but it also defines the following additional parameters
(in the given order):

\begin{table}[h]
\centering
\begin{tabular}{p{4.0cm}p{9.5cm}}
  \toprule
  \tt M1     & the lightest neutrino mass eigenstate (which becomes relevant
               for neutrino couplings to scalar DM) \\
  \midrule
  \tt M_DM   & the DM mass (currently unused) \\
  \midrule
  \tt ABS_CHI_$\alpha\beta$ \newline
      ARG_CHI_$\alpha\beta$
             & the absolute values and phases of the effective DM--neutrino
               couplings $\chi^{\alpha\eta}$ defined in \cref{eq:chi-dm}. The flavor
               indices $\alpha$ and $\beta$ can be {\tt E}, {\tt MU}, or
               {\tt TAU}.\\
  \midrule
  \tt ABS_XI_$i$, ARG_XI_$i$
             & the components of the DM polarization 4-vector $\xi$. The index
               $i$ runs from 0 to 3. Note that there is some redundancy in these
               parameters: for scalar DM, they are completely ignored, for
               unpolarized vector DM, only $|\vec\xi|$ matters, and for
               polarized vector DM, the physically relevant parameters
               are $|\vec\xi|$, $\xi^0$, and $\xi^3$. (We always assume the
               neutrino beam to be oriented along the $z$-axis.) \\
  \bottomrule
\end{tabular}
\end{table}

\noindent
It is strongly recommended to use {\tt glbSetParamName} to assign human-readable
names to the oscillation parameters by which they can be referred to in
subsequent calls to {\tt glbSetOscParamByName} and {\tt
glbSetProjectionFlagByName}.  The pre-defined names listed above can be retrieved
using
\begin{verbatim}
  const char *dm_get_param_name(const int i);
\end{verbatim}


\section{Making the Oscillation Engine Known to GLoBES}
%------------------------------------------------------

After initializing the fuzzy DM oscillation, we need to tell GLoBES that we want to
used it by calling
\begin{verbatim}
  glbRegisterProbabilityEngine(n_params,
                               &dm_probability_matrix,
                               &dm_set_oscillation_parameters,
                               &dm_get_oscillation_parameters,
                               NULL);
\end{verbatim}
The number of oscillation parameters {\tt n_params} in the first line can in principle
be deduced from the above table, but we highly advice not to hardcode it, but
to instead retrieve it by calling
\begin{verbatim}
  int dm_get_n_osc_params();
\end{verbatim}
before calling {\tt glbRegistareProbabilityEngine} (but after calling
{\tt dm_init_probability_engine}.

As with any new probability engine, the call to {\tt
glbRegisterProbabilityEngine} has to occur before any calls to {\tt
glbAllocParams} or {\tt glbAllocProjection} to make sure that all parameter and
projection vectors have the correct length.

Having initialized and registered the non-standard probability engine, we can
use all GLoBES functions in the same way as we would for standard
oscillations.

After using the fuzz DM oscillation engine, we may want to release the
(small) amount of memory allocated by it by calling
\begin{verbatim}
  dm_free_probability_engine();
\end{verbatim}


\section{Time Dependence}
%------------------------

Simulating the time dependence appearing in \cref{eq:phi-osc,eq:phi-osc-vector},
as well as a possible time dependence in the polarization vector $\xi^\mu$
coming from the motion of the Earth through the DM fluid, is not intrinsically
part of this package.  The reason is that it can be easily implemented by
varying $\phi_0$.

A possible strategy for including the time dependence is to run the GLoBES
simulation and fit multiple times for different time bins (with suitably
scaled {\tt @time} parameter in the flux definition). As the data in
the different time bins are independent, the total $\chi^2$ from such an
analysis can be obtained simply by adding the $\chi^2$'s from all time bins.

If running the full simulation for sufficiently many time steps is too
computationally expensive, the following approximation should work: use
only few time bins for the main simulation, but when analyzing it (for
instance in a user-defined GLoBES $\chi^2$ function), divide it into
finer time bins, interpolating between the coarse ones with a suitable
chosen oscillatory fit function (for instance a polynomial in $\cos(m_\phi t)$).
The parameters of the fit function typically do not need to be determined
separately for each energy bin because they very slowly with energy.  It
is instead sufficient to carry out the fit in only a few energy bins
and then interpolate the fit parameters in between.  If more accuracy in
the time domain is desired, one could imagine evaluating oscillation
probabilities with finer time binning for just those few bins.


\bibliographystyle{plain}
\bibliography{fuzzy-dm}


\end{document}

